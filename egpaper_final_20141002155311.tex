\documentclass[10pt,twocolumn,letterpaper]{article}

\usepackage{cvpr}
\usepackage{times}
\usepackage{epsfig}
\usepackage{graphicx}
\usepackage{amsmath}
\usepackage{amssymb}

% Include other packages here, before hyperref.

% If you comment hyperref and then uncomment it, you should delete
% egpaper.aux before re-running latex.  (Or just hit 'q' on the first latex
% run, let it finish, and you should be clear).
\usepackage[breaklinks=true,bookmarks=false]{hyperref}

\cvprfinalcopy % *** Uncomment this line for the final submission

\def\cvprPaperID{****} % *** Enter the CVPR Paper ID here
\def\httilde{\mbox{\tt\raisebox{-.5ex}{\symbol{126}}}}

% Pages are numbered in submission mode, and unnumbered in camera-ready
%\ifcvprfinal\pagestyle{empty}\fi
\setcounter{page}{4321}
\begin{document}

%%%%%%%%% TITLE
\title{Articulated Human Pose Estimation with Deep Convolutional Neural Networks}

\author{First Author\\
Institution1\\
Institution1 address\\
{\tt\small firstauthor@i1.org}
% For a paper whose authors are all at the same institution,
% omit the following lines up until the closing ``}''.
% Additional authors and addresses can be added with ``\and'',
% just like the second author.
% To save space, use either the email address or home page, not both
\and
Second Author\\
Institution2\\
First line of institution2 address\\
{\tt\small secondauthor@i2.org}
}

\maketitle
%\thispagestyle{empty}

%%%%%%%%% ABSTRACT
\begin{abstract}
\end{abstract}

%%%%%%%%% BODY TEXT
%-------------------------------------------------------------------------
\section{Introduction}
\begin{itemize}
  \item Pose Estimation popular and challenging
  \item Flexible Mixtures of Parts Model and HOG feature
  \item Deep Convolutional Neural Networks
  \item Deformable Part Models is Neural Network
\end{itemize}

%-------------------------------------------------------------------------
\section{Related Work}

%-------------------------------------------------------------------------
\section{Model} 

\subsection{Convolutional Part Detector}

\subsection{Higher Level Spatial Model}

\subsection{Unified Model}

\subsection{Deconvolutional Neural Networks}
\begin{itemize}
\item Fully Connected Layer
\item Max Pooling Layer
\item Convolutional Layer
\item Rectified Linear Layer
\item General Layer
\end{itemize}

\section{Inference}
\begin{itemize}
  \item Discrete activiation hidden variables
  \item Continuous acitivation hidden variables
  \item Iterativ Method for Inference
\end{itemize}

\section{Learning}
\begin{itemize}
  \item EM for discrete activiation hidden variables
  \item EM for continuous acitviation hidden variables
\end{itemize}

\subsection{Learning Unary Terms}

\subsection{Learning Pairwise Terms}

\subsection{Joint learning}

\subsection{Latent Update}

%-------------------------------------------------------------------------
\section{Experiments}

\subsection{Dataset}

\subsection{Experiment Details}
\begin{itemize}
  \item Only set feedback on RELU layer 
  \item Train multi-class feedback model with hidden variable sharing
\end{itemize}

\subsection{Visualization of feedbacks}
A figure shows Feedback vs No-Feedback visualization on fc8

Key story: Feedback surpress noise and extract salient part region and contexts

\subsection{Where to Add Feedback}
A figure shows Feedback vs No-Feedback visualization on conv5 etc. \\

Key story: 1. Not all layers need feedback 2. Feedback surpress noisefor high level layers

\subsection{Are the Activation Similar for different classes}
A figure shows Feedback visualization on fc8 for different classes \\

Key story: 1. Feedbacks are different for different classes in the same image. 2. Some similar classes share similar feedbacks

\subsection{Are feedback helpful for recognition?}

\subsection{Are feedback useful for localization}
A figure shows Feedback visualization for different object classes for the same image (a little similar to above)

\subsection{Are feedback robust to noise}
A figure shows Feedback visualization for an image producing a different class label than ground truth (similar as the \"intriguing property\" paper)

\subsection{How iterative update changes visualization}
A figure shows the visualization with update iteration

\subsection{Are feedback helpful for multi-task learning?}

%-------------------------------------------------------------------------

\begin{figure}[t]
\begin{center}
\fbox{\rule{0pt}{2in} \rule{0.9\linewidth}{0pt}}
   %\includegraphics[width=0.8\linewidth]{egfigure.eps}
\end{center}
   \caption{Example of caption.  It is set in Roman so that mathematics
   (always set in Roman: $B \sin A = A \sin B$) may be included without an
   ugly clash.}
\label{fig:long}
\label{fig:onecol}
\end{figure}

Finally, you may feel you need to tell the reader that more details can be
found elsewhere, and refer them to a technical report.  For conference
submissions, the paper must stand on its own, and not {\em require} the
reviewer to go to a techreport for further details.  Thus, you may say in
the body of the paper ``further details may be found
in~\cite{Authors14b}''.  Then submit the techreport as additional material.
Again, you may not assume the reviewers will read this material.

\begin{figure}[t]
\begin{center}
\fbox{\rule{0pt}{2in} \rule{0.9\linewidth}{0pt}}
   %\includegraphics[width=0.8\linewidth]{egfigure.eps}
\end{center}
   \caption{Example of caption.  It is set in Roman so that mathematics
   (always set in Roman: $B \sin A = A \sin B$) may be included without an
   ugly clash.}
\label{fig:long}
\label{fig:onecol}
\end{figure}

\begin{table}
\begin{center}
\begin{tabular}{|l|c|}
\hline
Method & Frobnability \\
\hline\hline
Theirs & Frumpy \\
Yours & Frobbly \\
Ours & Makes one's heart Frob\\
\hline
\end{tabular}
\end{center}
\caption{Results.   Ours is better.}
\end{table}

{\small
\bibliographystyle{ieee}
\bibliography{egbib}
}

\end{document}
